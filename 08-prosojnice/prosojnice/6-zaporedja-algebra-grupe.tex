\begin{frame}{Zaporedja, vrste in limite}
	\begin{enumerate}
		\item 
		Naj bo $\sum_{n=1}^{\infty}$ $a_n$ absolutno konvergentna vrsta in $a_n \ne -1$.
		Dokaži, da je tudi vrsta $\sum_{n=1}^\infty \frac{a_n}{1+a_n}$
		absolutno konvergentna.

		\item
		Izračunaj limito
		$$ \lim_{x\to\infty } (\sin\sqrt{x+1} -\sin\sqrt{x}). $$

		\item
		Za dani zaporedji preveri, ali sta konvergentni.
		% Pomagajte si s spodnjima delno pripravljenima matematičnima izrazoma:
		$$
		 a_n = \underbrace{t\sqrt{2+\sqrt{2+\dots+\sqrt{2}}}}_{n korenov} \qquad
		 b_n = \underbrace{\sin(\sin(\dots(\sin 1)\dots))}_{n sinusov}
		$$
		
	\end{enumerate}
\end{frame}

\begin{frame}{Algebra}
	\begin{enumerate}
		\item
		Vektorja $\vec{c}= \vec{a} + \vec{2b}$ in $\vec{d}= \vec{a}-\vec{b}$
		sta pravokotna in imata dolžino 1. Določi kot med vektorjema $\vec{a}$ in $\vec{b}$.
		\item 
		Izračunaj $
		{\begin{pmatrix}
			1 & 2 & 3 & 4 & 5 & 6\\
            4 & 5 & 2 & 6 & 3 & 1
		\end{pmatrix}}^{-2000} $
		
	\end{enumerate}
\end{frame}

\begin{frame}{Velika determinanta}     %\cdots  \vdots  \ldots  \ddots
	Izračunaj naslednjo determinanto $2n \times 2n$, ki ima na neoznačenih mestih ničle.
	$$\begin{vmatrix}
		1 &  & & & 1& & & & \\
		&  2 & & & 2 & & & &    \\
		& & & \ddots & & \vdots & & & &     \\
		& & & & n-1 & n-1 & & & &     \\
		1 & 2 & \cdots & n-1& n & n+1& n+2& \cdots  & & 2n      \\
		& & & & & n+1 & n+1 & & &     \\
		& & & & & n+2 & & n+2 & &     \\
		& & & & & \vdots& & & \ddots &     \\
		& & & & & 2n & & & & 2n     
    \end{vmatrix}$$
\end{frame}

\begin{frame}{Grupe}
	Naj bo
	??
	\begin{enumerate}
		\item
			Pokaži, da je $G$ podgrupa v grupi ??
			neničelnih kompleksnih števil za običajno množenje.
		\item
			Pokaži, da je $H$ podgrupa v aditivni grupi ??
			ravninskih vektorjev za običajno seštevanje po komponentah.
		\item
			Pokaži, da je preslikava $f:H\to G$, podana s pravilom
			??
			izomorfizem grup $G$ in $H$.
	\end{enumerate}
\end{frame}
